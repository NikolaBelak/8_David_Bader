\documentclass[a4paper]{article}

\usepackage{color}
\usepackage{url}
\usepackage[T2A]{fontenc} 
\usepackage[utf8]{inputenc} 
\usepackage{graphicx}

\usepackage[english,serbian]{babel}


\usepackage[unicode]{hyperref}
\hypersetup{colorlinks,citecolor=green,filecolor=green,linkcolor=blue,urlcolor=blue}


\newtheorem{primer}{Primer}[section]

\begin{document}

\title{David Bader\\ \small{Seminarski rad u okviru kursa\\Tehničko i naučno pisanje\\ Matematički fakultet}}
\author{Nikola Belaković, Vojkan Panić,\\ Filip Antonijević, Bogdan Damljanović\\ nidzoteri@gmail.com, vojkan.panic@gmail.com,\\ filipdantonijevic@gmail.com, bdamljanovic0@gmail.com }
\date{05.~novembar 2019.}
\maketitle

\abstract{
David A. Bader (rođen 4. maja 1969.) ugledni je profesor i direktor Instituta za nauku o podacima na Tehnološkom institutu u Nju Džersiju. Ranije je bio profesor, predsedavajući Škole računarske nauke i inženjerstva i izvršni direktor računarstva visokih performansi na računarskom fakultetu u Džordžiji. Pored toga, Bader je izabran za direktora prvog Soni Tošibinog centra za kompetenciju za mikroprocesor na Računarskom fakultetu na Tehnološkom institutu u Džordžiji. On je saradnik IEEE-a, AAAS-ov saradnik, SIAM-ov saradnik. Bader je vodeći stručnjak za nauke o podacima. Njegova glavna područja istraživanja nalaze se na preseku računarstava visokih performansi i aplikacija koje pomažu u svakodnevnom životu, uključujući bezbednost na internetu, masovnu analitiku i računsku genomiku.

Bader je stručnjak za dizajn i analizu paralelnih i višejezgarnih algoritama za aplikacije u stvarnom svetu, poput onih u bezbednosti na internetu i računskoj biologiji. Dobitnik je nagrade od IBM-a, Majkrosoftovih istraživanja, Envidia, Fejsbuk, Intel i Soni. Ko-predsedavao je nizom sastanaka. Bio je prepoznat kao jedan od najuticajnijih autora u istoriji međunarodne konferencije IEEE o računarima visokih performansi, podacima i analitikama (HiPC) u 2018.




\newpage



\bibliography{seminarski} 
\bibliographystyle{plain}

\end{document}
